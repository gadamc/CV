%------------------------------------
% Dario Taraborelli
% Typesetting your academic CV in LaTeX
%
% URL: http://nitens.org/taraborelli/cvtex
% DISCLAIMER: This template is provided for free and without any guarantee 
% that it will correctly compile on your system if you have a non-standard  
% configuration.
%------------------------------------


%!TEX TS-program = xelatex
%!TEX encoding = UTF-8 Unicode

\documentclass[11pt, a4paper]{article}
\usepackage{fontspec} 

% DOCUMENT LAYOUT
\usepackage{geometry} 
%\geometry{textwidth=5.5in, textheight=8.5in, marginparsep=7pt, marginparwidth=.6in}
% Europe
\geometry{textwidth=14cm, textheight=22cm, marginparsep=7pt, marginparwidth=.6in}
\setlength\parindent{0in}
\usepackage{enumitem}
\usepackage{datetime} 
\newdateformat{usvardate}{%
\THEDAY~\monthname[\THEMONTH]~\THEYEAR}
% FONTS
\defaultfontfeatures{Mapping=tex-text} % converts LaTeX specials (``quotes'' --- dashes etc.) to unicode
\setromanfont [Ligatures={Common}, Numbers={OldStyle}]{Hoefler Text}
\setmonofont[Scale=0.8]{Monaco} 
\setsansfont[Scale=0.9]{Optima Regular} 

% ---- CUSTOM COMANDS
% ---- CUSTOM AMPERSAND
\newcommand{\amper}{{\fontspec[Scale=.95]{Hoefler Text}\selectfont\itshape\&}}
% ---- MARGIN YEARS
\newcommand{\virginyears}[1]{\marginpar{\scriptsize #1}}
\newcommand{\years}[1]{\noindent\virginyears{#1}}
\newcommand{\MJ}      {{\sc{Majo\-ra\-na}}}
\newcommand{\TUM}      {Technische Universit\"{a}t M\"{u}nchen}
\def\minmod{{\sc Demonstrator}}
\newcommand{\cpp}{C\protect\raisebox{.25ex}{++}\ }
\newenvironment{packed_enum}{
\begin{itemize}[topsep=0pt]
  \renewcommand{\labelitemi}{$\cdot$}
  \setlength{\itemsep}{1pt}
  \setlength{\parskip}{0pt}
  \setlength{\parsep}{0pt}
}{\end{itemize}}
% ---- END CUSTOM COMANDS
% HEADINGS
\usepackage{sectsty} 
\usepackage[normalem]{ulem} 
\sectionfont{\sffamily\mdseries\scshape\large\sectionrule{0pt}{0pt}{-0.5ex}{0.5pt}} 
\subsectionfont{\rmfamily\mdseries\scshape\normalsize} 
\subsubsectionfont{\rmfamily\bfseries\upshape\normalsize} 

% PDF SETUP
% ---- FILL IN HERE THE DOC TITLE AND AUTHOR
\usepackage[unicode, bookmarks, pdfencoding=auto, colorlinks, breaklinks, pdftitle={Michael Marino - vita},pdfauthor={Michael Marino}]{hyperref}  
\hypersetup{linkcolor=blue,citecolor=blue,filecolor=black,urlcolor=blue} 

% DOCUMENT
\begin{document}

%Add some length after paragraphs to improve readability
\setlength{\parskip}{1ex}

%
% Add the following command to grab *everything* from the bib database
\nocite{*}
%

\reversemarginpar
\textsf{\LARGE Michael G.~Marino}\\[1cm]
Excellence Cluster `Universe'\\
Physik-Department\\
\TUM\\
Boltzmannstr. 2\\
80805 Munich\\
Germany\\[.2cm]
Phone: \texttt{+49 89 35831 7149}\\
Mobile: \texttt{+49 176 66830986}\\[.2cm]
\def\myemail{michael.marino@mytum.de}
email: \href{mailto:\myemail}{\myemail}\\
%\vfill

%%\hrule
\section*{Education}
\years{2010}\textsc{PhD} in Physics, University of Washington, Seattle, Washington

\years{2005}\textsc{MSc} in Physics, University of Washington, Seattle, Washington

\years{2004}\textsc{BS} in Physics, \emph{Summa cum laude}, University of Notre Dame, Notre Dame, Indiana

\years{2004}\textsc{BA} in Philosophy, \emph{Summa cum laude}, University of Notre Dame, Notre Dame, Indiana

%%\hrule
\section*{Awards \amper{} honors}

\years{2007}Sebastian Karrer Prize, Physics Department, University of Washington

\years{2006}DNP Travel Award, APS Meeting, Dallas, Texas

\years{2005}Sebastian Karrer Scholarship, Physics Department, University of Washington

\years{2004-2005}Mellam Fellowship, Physics Department, University of Washington

\years{2004}Outstanding Senior Physics Major Award, Physics Department, University of Notre Dame

\years{2004}Elected Member, Phi Beta Kappa, Notre Dame Chapter, Epsilon of Indiana

% new page to deal with funky page break
\newpage
% Publication/Presentations
\section*{Publications \amper{} Presentations}
\subsection*{Journal articles}
% Journal articles, parse everything from bib database and order correctly
\bibliographystyle{mgm_cv}
\bibliography{cv_mgm}

\subsection*{Presentations}

\years{2010}``Searching for low-mass WIMPs: Dark matter from the tabletop", E-18 Seminar, \TUM, Garching, Germany. \emph{Invited Talk}

\years{2010}``Searching for low-mass WIMPs: Dark matter from the tabletop", Research Progress Seminar, Physics Division, Lawrence Berkeley National Laboratory, Berkeley, California. \emph{Invited Talk}

\years{2010}``Searching for low-mass WIMPs: Dark matter from the tabletop", Max-Planck-Institut f\"{u}r Physik, Munich, Germany \emph{Invited Talk}

\years{2010}``Searching for low-mass WIMPs: Dark matter from the tabletop", Max-Planck-Institut f\"{u}r Kernphysik, Heidelberg, Germany \emph{Invited Talk}

\years{2010}``Dark matter searches with low-noise ionization germanium detectors", Workshop on Germanium-Based Detectors and Technologies, Berkeley, California. \emph{Invited Talk}

\years{2010}``P-type point contact detectors for the \MJ~experiment", Los Alamos National Lab, Los Alamos, New Mexico. \emph{Invited Talk}

\years{2009}``The \MJ~neutrinoless double-beta decay experiment", DPF Meeting, Detroit, Michigan. \emph{Talk}

\years{2008}``Novel germanium detectors for the \MJ~experiment”, National Nuclear Physics Summer School, George Washington University, Washington, D.C.  \emph{Talk}

\years{2007}``Validation of neutron transportation and production in Geant4”, TRIUMF Summer Institute, Vancouver, British Columbia.  \emph{Talk}

\years{2006}``Implementation of a generic surface sampler using Geant4”, IEEE Nuclear Science Symposium, San Diego, California.  \emph{Poster}

\years{2006}``The proposed \MJ~$0\nu\beta\beta$ experiment”, Neutrino Conference 2006, San\-ta Fe, New Mexico.  \emph{Poster}

\years{2006}``An update on the \MJ-GERDA simulation package (MaGe)”, APS Meeting, Dallas, Texas.  \emph{Talk}

\years{2003}``ZFC/FC Simulations of Nano-clusters in Metallic Matrices”, Research Experience for Undergraduates Colloquium, University of Idaho.  \emph{Talk}

\years{2002}``Monte Carlo Analysis of UCN Transport Properties”, Research Experience for Undergraduates/Research Experience for Teachers Colloquium, University of Notre Dame.  \emph{Talk}

% new page to deal with funky page break
%\newpage
% Professional experience
\section*{Relevant Professional Experience}
\years{2010-present} \emph{Postdoktorand}, EXC `Universe', \TUM, Munich, Germany
\begin{packed_enum}
	\item Member of the EXO~collaboration (Analysis and Simulation software development)
	\item Member of the TUM nEDM experiment (DAQ hardware and software development)
\end{packed_enum}	
\years{2005-2010}\emph{Graduate Research Assistant}, CENPA, University of Washington, Seattle, Washington\\ Advisor: John F. Wilkerson
\begin{packed_enum}
	\item Member of the \MJ~collaboration
	\item Member of the CoGeNT~collaboration	
	\item Software: (i) neutron simulations to determine systematic errors of neutron background estimates, (ii) primary and collaborative contribution to the design, implementation and testing of several analysis software packages for the \MJ~experiment, including a modular framework for pulse-shape analysis, an encapsulation package for serializing and storing data and metadata, and a ROOT-based package for processing ORCA binary files
	\item Hardware: (i) design, development and testing of digitizers within the ORCA DAQ program (ii) design, development, and testing of software framework for embedded processors for the ORCA program, including low-level Linux driver design and implementation, (iii) full design, testing, and deployment of a fully digital, ORCA-based DAQ for a p-type point-contact germanium detector at Soudan Underground Lab.
	\item Analysis: (i) determination of fast digitizer requirements for the \MJ~experiment, (ii) full design and construction of an end-to-end solution for data flow and analysis for remote P-PC detector deployed at Soudan Underground Lab, (iii) calculation of relevant exclusion limits for dark matter for low-background, low-threshold germanium detectors.
\end{packed_enum}

\years{2003}{\emph{Summer Research Assistant}, NSF Research Experience for Undergraduates Program, University of Idaho, Moscow, Idaho}\\ {Advisor: You Qiang}
\begin{packed_enum}
\item Contributed to research concerning the magnetic properties of materials created using nano-cluster deposition techniques
\item Performed Monte Carlo calculations to simulate Zero-Field Cooled/Field-Cooled (ZFC/FC) magnetization measurements of materials composed of nano-clusters in different matrices (metallic and non-metallic)
\item Worked with nano-cluster deposition apparatus, helping to create samples of varying characteristics through different deposition techniques
\end{packed_enum}

\years{2002}{\emph{Summer Research Assistant}, NSF Research Experience for Undergraduates Program, University of Notre Dame, Notre Dame, Indiana}\\{ Advisor: Alejandro Garcia}
\begin{packed_enum}
\item Contributed to research concerning the measurement of the asymmetry of beta-decay of neutrons
\item Performed Monte Carlo calculations analyzing the transport properties of ultra-cold neutrons (UCN) through guide pipes
\end{packed_enum}

% new page to deal with funky page break
%\newpage
\years{2001-2002}{\emph{Tutor}, Learning Resource Center, University of Notre Dame, Notre Dame, Indiana}
\begin{packed_enum}
\item Tutored individual first-year physics majors
\item Led group collaborative learning sessions in physics for science and engineering majors
\end{packed_enum}

\years{2001-2002}{\emph{Lab Assistant}, Nuclear Structure Laboratory, University of Notre Dame, Notre Dame, Indiana}\\ {Advisor: Larry Lamm }
\begin{packed_enum}
\item Trained to run FN Tandem Accelerator ($10$~MV)
\item Machined parts for research groups (mill experience, lathe experience, soldering experience)
\item Maintained equipment in the laboratory (roughing pumps, cryostats, etc.)
\item Assembled and maintained portions of beam line and related vacuum for various research projects
\end{packed_enum}

% Outreach, not necessarily physics-based
\section*{Outreach Experience}
\years{2008-2009}\emph{President}, Career Development Organization of Physicists and Astronomers, University of Washington, Seattle, Washington
\begin{packed_enum}
\item Student-run organization focused on providing career resources for graduate students inside and outside of academia  
\item Planned and executed flagship event, 2008 Networking Day.  Networking Day provides a forum to students to present their research to interested employers, obtain ideas for future careers.
\item Planned career development workshops for students
\end{packed_enum}

% Software and computing
\section*{Software and Computing}
\begin{packed_enum}
\item Fluent in Fortran (simulation, theoretical calculations), C (Linux kernel, DAQ software, simulation), \cpp (simulation, analysis, DAQ software), Obj-C (DAQ software), Scripting languages (Python, Bash, Tcsh), and debugging software (gdb, valgrind, kdb)
\item Fluent in the software packages: ROOT, Geant4, CLHEP
\item General experience with Perl, Math\-ematica, Databases (SQL, CouchDB), Ja\-va\-script, HTML, XML, Qt
%\vspace{1cm}
\end{packed_enum}
\vfill{}
\hrulefill

% FILL IN THE FULL URL TO YOUR CV
\begin{center}
{\footnotesize \href{http://dl.dropbox.com/u/979314/cv_mgm.pdf}{http://dl.dropbox.com/u/979314/cv\_mgm.pdf} — Last updated: \usvardate\today
}
\end{center}


\end{document}
